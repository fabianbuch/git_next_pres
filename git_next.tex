
\documentclass{beamer}

\mode<presentation>
{
  \usetheme{Malmoe}
  \usecolortheme{dolphin}
  \setbeamercovered{transparent}
}


%%% ngerman: language set to new-german
%\usepackage{german}
\usepackage[ngerman,english]{babel}
%%% inputenc: coding of german special characters
\usepackage[utf8]{inputenc}
%%% fontenc, ae, aecompl: coding of characters in PDF documents
\usepackage[T1]{fontenc}
\usepackage{ae,aecompl}
\usepackage{times}

\title[] % (optional, nur bei langen Titeln nötig)
{Git}

\subtitle
{Next Generation Version Control}

\author[] % (optional, nur bei vielen Autoren)
{Fabian Buch}
% - Namen müssen in derselben Reihenfolge wie im Papier erscheinen.
% - Der \inst{?} Befehl sollte nur verwendet werden, wenn die Autoren
%   unterschiedlichen Instituten angehören.

\institute[] % (optional, aber oft nötig)
{Synyx GmbH \& Co. KG}
% - Der \inst{?} Befehl sollte nur verwendet werden, wenn die Autoren
%   unterschiedlichen Instituten angehören.
% - Keep it simple, niemand interessiert sich für die genau Adresse.

\date[] % (optional, sollte der abgekürzte Konferenzname sein)
{\today}
% - Voller oder abgekürzter Name sind möglich.
% - Dieser Eintrag ist nicht für das Publikum gedacht (das weiß
%   nämlich, bei welcher Konferenz es ist), sondern für Leute, die die
%   Folien später lesen.

\subject{Git - Next Generation Version Control}
% Dies wird lediglich in den PDF Informationskatalog einfügt. Kann gut
% weggelassen werden.


% Falls eine Logodatei namens "university-logo-filename.xxx" vorhanden
% ist, wobei xxx ein von latex bzw. pdflatex lesbares Graphikformat
% ist, so kann man wie folgt ein Logo einfügen:

%\pgfdeclareimage[height=0.4cm]{logo-hs}{logo_hs}
%\logo{\pgfuseimage{logo-hs}}

%\newcommand\BackgroundPicture[3]{
%    \setbeamertemplate{background}{
%        \parbox[c][\paperheight]{\paperwidth}{
%            \vfill \hfill
%            \includegraphics[width=#2\paperwidth,height=#3\paperheight]{#1}
%            \hfill \vfill
%        }
%    }
%}

%\usebackgroundtemplate{
%\includegraphics[width=\paperwidth,
%height=\paperheight]{synyxbg}
%}

%\pgfdeclareimage[height=20cm]{synyxbg}{synyxbg}
%\logo{\pgfuseimage{synyxbg}}

\begin{document}
%\setbeamertemplate{background canvas}{\includegraphics[height=262px,width=\paperwidth]{praesentationshg_ohne}}
%\BackgroundPicture{synyxbg,1,1}



\begin{frame}
  \titlepage
\end{frame}

%\begin{frame}{Agenda}
%  \tableofcontents[pausesections]
%  % Die Option [pausesections] könnte nützlich sein.
%\end{frame}



% Einen Vortrag zu strukturieren ist nicht immer einfach. Die
% nachfolgende Struktur kann unangemessen sein. Hier ein paar Regeln,
% die für diese Lösungsvorlage gelten:

% - Es sollte genau zwei oder drei Abschnitte geben (neben der
%   Zusammenfassung). 
% - *Höchstens* drei Unterabschnitte pro Abschnitt.
% - Pro Rahmen sollte man zwischen 30s und 2min reden. Es sollte also
%   15 bis 30 Rahmen geben.

% - Konferenzteilnehmer wissen oft wenig von der Materie des
%   Vortrags. Deshalb: vereinfachen!
% - In 20 Minuten ist es schon schwer genug, die Hauptbotschaft zu
%   vermitteln. Deshalb sollten Details ausgelassen werden, selbst
%   wenn dies zu Ungenauigkeiten oder Halbwahrheiten führt.          
% - Falls man Details weglässt, die eigentlich wichtig für einen
%   Beweis/Implementation sind, so sagt man dies einmal nüchtern. Alle
%   werden damit glücklich sein.

\section{Git Introduction}

\subsection{Introduction to Git}\label{sub:einleitung}


%\begin{frame}{Überschriften müssen informativ sein.\\
%    Korrekte Groß-/Kleinschreibung beachten.}{Untertitel sind optional.}
  % - Eine Überschrift fasst einen Rahmen verständlich zusammen. Man
  %   muss sie verstehen können, selbst wenn man nicht den Rest des
  %   Rahmens versteht.


\begin{frame}{Introduction to Git}
  \begin{itemize}
  \item
    Basic theory and hands on tips
  \end{itemize}
\end{frame}

\section{Version Control}

\subsection{History}

\begin{frame}{How did we get to what we call Version Control today?}
\end{frame}

\subsection{Local Version Control}

\begin{frame}{Manual}
  \begin{itemize}
  \item
    numbered directories
  \item
    timestamping
  \end{itemize}
\end{frame}

\begin{frame}{Local Version Control (e.g. RCS)}
    \pgfdeclareimage[height=5cm]{local computer}{img/18333fig0101-tn}
    \centering
    \pgfuseimage{local computer}
    \hfill\vfill
\end{frame}

\subsection{Centralized Version Control}

\begin{frame}{Centralized Version Control}
    \pgfdeclareimage[height=5cm]{local computer}{img/18333fig0102-tn}
    \centering
    \pgfuseimage{local computer}
    \hfill\vfill
\end{frame}

\begin{frame}{Centralized Version Control Systems}
  \begin{itemize}
  \item
    CVS
  \item
    Subversion
  \item
    Perforce
  \end{itemize}
\end{frame}


\subsection{Distributed Version Control}

\begin{frame}{Distributed Version Control}
    \pgfdeclareimage[height=5cm]{local computer}{img/18333fig0103-tn}
    \centering
    \pgfuseimage{local computer}
    \hfill\vfill
\end{frame}

\begin{frame}{Distributed Version Control Systems}
  \begin{itemize}
  \item
    Git
  \item
    Gnu Arch
  \item
    Darcs
  \end{itemize}
\end{frame}

\section{Git Basics}

\subsection{Historic Goals}

\begin{frame}{Historic Goals}
  \begin{itemize}
  \item
    Speed
  \item
    Simple design
  \item
    Strong support for non-linear development (thousands of parallel branches)
  \item
    Fully distributed
  \item
    Able to handle large projects like the Linux kernel efficiently (speed and data size)
  \end{itemize}
\end{frame}

\subsection{Snapshots, Not Differences}

\begin{frame}{Deltas}
    \pgfdeclareimage[height=5cm]{local computer}{img/18333fig0104-tn}
    \centering
    \pgfuseimage{local computer}
    \hfill\vfill
\end{frame}

\begin{frame}{Snapshots}
    \pgfdeclareimage[height=5cm]{local computer}{img/18333fig0105-tn}
    \centering
    \pgfuseimage{local computer}
    \hfill\vfill
\end{frame}

\subsection{Working Offline}

\begin{frame}{Working Offline}
  \begin{itemize}
  \item
    Nearly all operations are local
  \item
    full local copy of repository (full history)
  \end{itemize}
\end{frame}

\subsection{Git has Integrity}

\begin{frame}{Git has Integrity}
  \begin{itemize}
  \item
    Everything is SHA-1 check-summed
  \end{itemize}
\end{frame}

\subsection{The three States}

\begin{frame}{The three States}
  \begin{itemize}
  \item
    commited
  \item
    modified
  \item
    staged
  \end{itemize}
\end{frame}

\begin{frame}{The three States}
    \pgfdeclareimage[height=5cm]{local computer}{img/18333fig0106-tn}
    \centering
    \pgfuseimage{local computer}
    \hfill\vfill
\end{frame}

\begin{frame}{File Status Lifecycle}
    \pgfdeclareimage[height=5cm]{local computer}{img/18333fig0201-tn}
    \centering
    \pgfuseimage{local computer}
    \hfill\vfill
\end{frame}

\subsection{Demo}

\begin{frame}{Demo}
 init, add, commit, status, file staged and unstaged, git diff, git diff --cached (--staged), git log --graph, gitk, git checkout -- <file>
\end{frame}

\section{Branching}

\subsection{What is a Branch in Git?}

\begin{frame}{Single commit repository data.}
    \pgfdeclareimage[height=5cm]{local computer}{img/18333fig0301-tn}
    \centering
    \pgfuseimage{local computer}
    \hfill\vfill
\end{frame}

\begin{frame}{Git object data for multiple commits.}
    \pgfdeclareimage[height=5cm]{local computer}{img/18333fig0302-tn}
    \centering
    \pgfuseimage{local computer}
    \hfill\vfill
\end{frame}

\begin{frame}{Branch pointing into the commit data's history.}
    \pgfdeclareimage[height=5cm]{local computer}{img/18333fig0303-tn}
    \centering
    \pgfuseimage{local computer}
    \hfill\vfill
\end{frame}

\begin{frame}{Create a new branch}
  \$ git branch testing
\end{frame}

\begin{frame}{Multiple branches pointing into the commit's data history.}
    \pgfdeclareimage[height=5cm]{local computer}{img/18333fig0304-tn}
    \centering
    \pgfuseimage{local computer}
    \hfill\vfill
\end{frame}

\begin{frame}{HEAD file pointing to the branch you're on.}
    \pgfdeclareimage[height=5cm]{local computer}{img/18333fig0305-tn}
    \centering
    \pgfuseimage{local computer}
    \hfill\vfill
\end{frame}

\begin{frame}{Switch branch}
  \$ git checkout testing
\end{frame}

\begin{frame}{HEAD points to another branch when you switch branches.}
    \pgfdeclareimage[height=5cm]{local computer}{img/18333fig0306-tn}
    \centering
    \pgfuseimage{local computer}
    \hfill\vfill
\end{frame}

\begin{frame}{Do some changes}
  \$ vim test.rb\\
  \$ git commit -a -m 'made a change'
\end{frame}

\begin{frame}{The branch that HEAD points to moves forward with each commit.}
    \pgfdeclareimage[height=5cm]{local computer}{img/18333fig0307-tn}
    \centering
    \pgfuseimage{local computer}
    \hfill\vfill
\end{frame}

\begin{frame}{Switch branch to master}
  \$ git checkout master
\end{frame}

\begin{frame}{HEAD moves to another branch on a checkout.}
    \pgfdeclareimage[height=5cm]{local computer}{img/18333fig0308-tn}
    \centering
    \pgfuseimage{local computer}
    \hfill\vfill
\end{frame}

\begin{frame}{Do some other changes}
  \$ vim test.rb\\
  \$ git commit -a -m 'made other change'
\end{frame}

\begin{frame}{The branch histories have diverged.}
    \pgfdeclareimage[height=5cm]{local computer}{img/18333fig0309-tn}
    \centering
    \pgfuseimage{local computer}
    \hfill\vfill
\end{frame}

\subsection{Merging}

\begin{frame}{A Hotfix}
  \$ git checkout -b 'hotfix'\\
  Switched to a new branch "hotfix"\\
  \$ vim index.html\\
  \$ git commit -a -m 'fixed the broken email address'\\
  hotfix: created 3a0874c: "fixed the broken email address"\\
   1 files changed, 0 insertions(+), 1 deletions(-)
\end{frame}

\begin{frame}{Hotfix Branch}
    \pgfdeclareimage[height=5cm]{local computer}{img/18333fig0313-tn}
    \centering
    \pgfuseimage{local computer}
    \hfill\vfill
\end{frame}

\begin{frame}{Fast forward merge}
  \$ git checkout master\\
  \$ git merge hotfix\\
  Updating f42c576..3a0874c\\
  Fast forward\\
   README |    1 -\\
   1 files changed, 0 insertions(+), 1 deletions(-)
\end{frame}


\begin{frame}{Your master branch points to the same place as your hotfix branch after the merge.}
    \pgfdeclareimage[height=5cm]{local computer}{img/18333fig0314-tn}
    \centering
    \pgfuseimage{local computer}
    \hfill\vfill
\end{frame}

\begin{frame}{Delete hotfix branch}
  \$ git branch -d hotfix\\
  Deleted branch hotfix (3a0874c).
\end{frame}

\begin{frame}{back to your work-in-progress branch}
  \$ git checkout iss53\\
  Switched to branch "iss53"\\
  \$ vim index.html\\
  \$ git commit -a -m 'finished the new footer  issue 53 '\\
  iss53: created ad82d7a: "finished the new footer  issue 53 "\\
   1 files changed, 1 insertions(+), 0 deletions(-)
\end{frame}

\begin{frame}{Your iss53 branch can move forward independently.}
    \pgfdeclareimage[height=5cm]{local computer}{img/18333fig0315-tn}
    \centering
    \pgfuseimage{local computer}
    \hfill\vfill
\end{frame}

\begin{frame}{Merge finished feature into master}
  \$ git checkout master\\
  \$ git merge iss53\\
  Merge made by recursive.\\
   README |    1 +\\
   1 files changed, 1 insertions(+), 0 deletions(-)
\end{frame}

\begin{frame}{Git automatically identifies the best common-ancestor merge base for branch merging.}
    \pgfdeclareimage[height=5cm]{local computer}{img/18333fig0316-tn}
    \centering
    \pgfuseimage{local computer}
    \hfill\vfill
\end{frame}

\begin{frame}{Git automatically creates a new commit object that contains the merged work.}
    \pgfdeclareimage[height=5cm]{local computer}{img/18333fig0317-tn}
    \centering
    \pgfuseimage{local computer}
    \hfill\vfill
\end{frame}

\begin{frame}{Up Next}
  \begin{itemize}
  \item
    Distributed Git
  \item
    Changing History
  \item
    DOs and DON'Ts
  \end{itemize}
\end{frame}


\end{document}

