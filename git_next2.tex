
\documentclass{beamer}

\mode<presentation>
{
  \usetheme{Malmoe}
  \usecolortheme{dolphin}
  \setbeamercovered{transparent}
}


%%% ngerman: language set to new-german
%\usepackage{german}
\usepackage[ngerman]{babel}
%%% inputenc: coding of german special characters
\usepackage[utf8]{inputenc}
%%% fontenc, ae, aecompl: coding of characters in PDF documents
\usepackage[T1]{fontenc}
\usepackage{ae,aecompl}
\usepackage{times}

\title[] % (optional, nur bei langen Titeln nötig)
{Git}

\subtitle
{Digging deeper}

\author[] % (optional, nur bei vielen Autoren)
{Fabian Buch}
% - Namen müssen in derselben Reihenfolge wie im Papier erscheinen.
% - Der \inst{?} Befehl sollte nur verwendet werden, wenn die Autoren
%   unterschiedlichen Instituten angehören.

\institute[] % (optional, aber oft nötig)
{Synyx GmbH \& Co. KG}
% - Der \inst{?} Befehl sollte nur verwendet werden, wenn die Autoren
%   unterschiedlichen Instituten angehören.
% - Keep it simple, niemand interessiert sich für die genau Adresse.

\date[] % (optional, sollte der abgekürzte Konferenzname sein)
{2010-12-10}
% - Voller oder abgekürzter Name sind möglich.
% - Dieser Eintrag ist nicht für das Publikum gedacht (das weiß
%   nämlich, bei welcher Konferenz es ist), sondern für Leute, die die
%   Folien später lesen.

\subject{}
% Dies wird lediglich in den PDF Informationskatalog einfügt. Kann gut
% weggelassen werden.


% Falls eine Logodatei namens "university-logo-filename.xxx" vorhanden
% ist, wobei xxx ein von latex bzw. pdflatex lesbares Graphikformat
% ist, so kann man wie folgt ein Logo einfügen:

%\pgfdeclareimage[height=0.4cm]{logo-hs}{logo_hs}
%\logo{\pgfuseimage{logo-hs}}

%\newcommand\BackgroundPicture[3]{
%    \setbeamertemplate{background}{
%        \parbox[c][\paperheight]{\paperwidth}{
%            \vfill \hfill
%            \includegraphics[width=#2\paperwidth,height=#3\paperheight]{#1}
%            \hfill \vfill
%        }
%    }
%}

%\usebackgroundtemplate{
%\includegraphics[width=\paperwidth,
%height=\paperheight]{synyxbg}
%}

%\pgfdeclareimage[height=20cm]{synyxbg}{synyxbg}
%\logo{\pgfuseimage{synyxbg}}

\begin{document}
%\setbeamertemplate{background canvas}{\includegraphics[height=262px,width=\paperwidth]{praesentationshg_ohne}}
%\BackgroundPicture{synyxbg,1,1}



\begin{frame}
  \titlepage
\end{frame}

%\begin{frame}{Agenda}
%  \tableofcontents[pausesections]
%  % Die Option [pausesections] könnte nützlich sein.
%\end{frame}



% Einen Vortrag zu strukturieren ist nicht immer einfach. Die
% nachfolgende Struktur kann unangemessen sein. Hier ein paar Regeln,
% die für diese Lösungsvorlage gelten:

% - Es sollte genau zwei oder drei Abschnitte geben (neben der
%   Zusammenfassung). 
% - *Höchstens* drei Unterabschnitte pro Abschnitt.
% - Pro Rahmen sollte man zwischen 30s und 2min reden. Es sollte also
%   15 bis 30 Rahmen geben.

% - Konferenzteilnehmer wissen oft wenig von der Materie des
%   Vortrags. Deshalb: vereinfachen!
% - In 20 Minuten ist es schon schwer genug, die Hauptbotschaft zu
%   vermitteln. Deshalb sollten Details ausgelassen werden, selbst
%   wenn dies zu Ungenauigkeiten oder Halbwahrheiten führt.          
% - Falls man Details weglässt, die eigentlich wichtig für einen
%   Beweis/Implementation sind, so sagt man dies einmal nüchtern. Alle
%   werden damit glücklich sein.

\section{Tags}

\begin{frame}{Tags}
  \begin{itemize}
  \item
    lightweight tags
  \item
    tag objects
  \item
    gpg signed tags (useful for releases)
  \end{itemize}
\end{frame}

\section{Distributed Git}

\subsection{Distributed Workflows}

\begin{frame}{Distributed Workflows}
  \begin{itemize}
  \item
    Unlike Centralized Version Control Systems (CVCSs), the distributed nature of Git allows you to be far more flexible in how developers collaborate on projects.
  \end{itemize}
\end{frame}

\begin{frame}{Centralized Workflow}
  \begin{itemize}
  \item
    One central hub, or repository, can accept code, and everyone synchronizes their work to it.
  \end{itemize}
\end{frame}

\begin{frame}{Centralized Workflow}
    \pgfdeclareimage[height=5cm]{centralized}{img/18333fig0501-tn}
    \centering
    \pgfuseimage{centralized}
    \hfill\vfill
\end{frame}

\begin{frame}{Integration Manager Workflow}
    \pgfdeclareimage[height=5cm]{github}{img/18333fig0502-tn}
    \centering
    \pgfuseimage{github}
    \hfill\vfill
\end{frame}

\begin{frame}{Integration Manager Workflow - Advantages}
  \begin{itemize}
  \item
    you can continue to work
  \item
    the maintainer of the main repository can pull in your changes at any time
  \item
    contributors don't have to wait for the project to incorporate their changes
  \item
    each party can work at their own pace
  \end{itemize}
\end{frame}

\begin{frame}{Dictator and Lieutenants Workflow}
  \begin{itemize}
  \item
    a variant of a multiple-repository workflow
  \item
    generally used by huge projects with hundreds of collaborators
  \item
    various integration managers are in charge of certain parts of the repository
  \end{itemize}
\end{frame}

\begin{frame}{Dictator and Lieutenants Workflow}
    \pgfdeclareimage[height=5cm]{dictator}{img/18333fig0503-tn}
    \centering
    \pgfuseimage{dictator}
    \hfill\vfill
\end{frame}

\subsection{Remote Branches}

\begin{frame}{Remote Branches}
  \begin{itemize}
    \item
      remote branches are references to the state of branches on your remote repositories
    \item
      they're local branches that you can't move
    \item
      they're moved automatically whenever you do any network communication
    \item
      (remote)/(branch)
  \end{itemize}
\end{frame}

\subsection{Demo}

\begin{frame}{Demo}
  \begin{itemize}
    \item
      empty central repo --bare
    \item
      second user: git clone
    \item
      git add remote origin
    \item
      git fetch
    \item
      git merge
    \item
      git pull
    \item
      git push
    \item
      git format-patch origin
    \item
  \end{itemize}
\end{frame}

\section{Changing History}

\subsection{Don't Change History available to Others}

\begin{frame}{Don't Change History available to Others}
  \begin{itemize}
    \item
      !!
    \item
      unless you know exactly what you're doing
  \end{itemize}
\end{frame}

\subsection{Cherry Pick (doesn't change history)}

\begin{frame}{Cherry Pick (doesn't change history)}
    \pgfdeclareimage[height=5cm]{cherrypick1}{img/18333fig0526-tn}
    \centering
    \pgfuseimage{cherrypick1}
    \hfill\vfill
\end{frame}

\begin{frame}{Cherry Pick (doesn't change history)}
  git cherry-pick e43a6
\end{frame}

\begin{frame}{Cherry Pick (doesn't change history)}
    \pgfdeclareimage[height=5cm]{cherrypick2}{img/18333fig0527-tn}
    \centering
    \pgfuseimage{cherrypick2}
    \hfill\vfill
\end{frame}

\begin{frame}{Cherry Pick (doesn't change history)}
  git branch -d idea-branch
\end{frame}

\subsection{--amend}

\begin{frame}{--amend}
  \begin{itemize}
    \item
      git commit --amend
    \item
      opens last commit for changes (commit msg, files..)
  \end{itemize}
\end{frame}

\subsection{Rebase}

\begin{frame}{Rebase - before}
    \pgfdeclareimage[height=5cm]{rebase1}{img/18333fig0516-tn}
    \centering
    \pgfuseimage{rebase1}
    \hfill\vfill
\end{frame}

\begin{frame}{Rebase}
  \$ git checkout featureA\\
  \$ git rebase origin/master\\
\end{frame}

\begin{frame}{Rebase - after}
    \pgfdeclareimage[height=5cm]{rebase1}{img/18333fig0517-tn}
    \centering
    \pgfuseimage{rebase1}
    \hfill\vfill
\end{frame}

\section{DOs and DON'Ts}

\begin{frame}{DOs and DON'Ts}
\end{frame}


\section{Tips and Tricks}

\begin{frame}{Tips and Tricks / Demo}
  \begin{itemize}
    \item
      first line of commit msg is used as summary
    \item
      git diff --check
    \item
      git log featureXY --not master
    \item
      git shortlog --no-merges master --not v1.0.1
    \item
      git reflog
    \item
      git show master@{yesterday}
  \end{itemize}
\end{frame}


\begin{frame}{Git SVN}
  \begin{itemize}
    \item
      centralized workflow
    \item
      linear history
    \item
      git stash
  \end{itemize}
\end{frame}


\section{Up Next}

\begin{frame}{Up Next}
  \begin{itemize}
    \item
      advanced topics
    \item
      fast-forward merges of feature-branches?
    \item
      advanced branching strategies
  \end{itemize}
\end{frame}


\end{document}

